\chapter{Etude des objectifs}

Cette partie du projet consiste à mettre en forme les fichiers obtenus via le parseur et à développer des fonctionnalités permettant à l'utilisateur de facilement utiliser la documentation.\par\bigskip

On peut séparer l'ensemble des tâches de cette section en 2 parties :
\begin{itemize}
\item Affichage d'une documentation épurée
\item Utilisation du code source
\end{itemize}\par\bigskip

Une majeure partie du travail demandé consiste à "augmenter" du code HTML avec des fonctionnalités JavaScript. L'essentiel du travail de cette partie est donc réalisée avec ce langage.\par

Cependant, une autre partie (décrite précédemment) consiste à mettre en place une syntaxe des fichiers HTML afin de faciliter l'intégration des deux modules Parseur et Web.

\section{Technologies utilisées}

Afin d'obtenir un développement efficace, plusieurs librairies sont utilisées :
\begin{description}
  \item[JQuery] Lib JS - Permet de manipuler des documents HTML plus simplement en JavaScript
  \item[Bootstrap] Framework CSS - Permet d'utiliser une grille pour placer des élements HTML 
  \item[MathJax] Lib JS - Permet d'afficher des formules mathématiques à partir de formules LaTeX, MathML, etc.
\end{description}