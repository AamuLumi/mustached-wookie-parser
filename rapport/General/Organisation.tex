\chapter{Organisation}

\section{Outil de versionnage}

Afin d'avoir des sauvegardes régulières de nos travaux et de pouvoir partager notre projet, nous utilisons un répertoire Git disponible sur \href{https://github.com}{GitHub} à l'adresse suivante : \href{https://github.com/AamuLumi/mustached-wookie-parser}{https://github.com/AamuLumi/mustached-wookie-parser}

\section{Développement modulaire}

Le projet est le développement de deux modules :
\begin{itemize}
\item le module de traitement des fichiers C
\item le module d'affichage de la documentation
\end{itemize}\par\bigskip
L'avantage d'un développement modulaire est qu'il est possible d'avancer sur plusieurs tâches à la fois : un module n'a pas besoin d'attendre l'autre pour pouvoir fonctionner.\par 
Cependant, il faut veiller à avoir une bonne communication (sur le formatage des fichiers, la syntaxe à utiliser, etc.) afin d'avoir une intégration finale quasi-instantanée.

\section{Répartition des tâches}